%%% LaTeX Template
%%% This template is made for project reports
%%%	You may adjust it to your own needs/purposes
%%%
%%% Copyright: http://www.howtotex.com/
%%% Date: March 2011

%%% Preamble
\documentclass[paper=a4, fontsize=11pt]{scrartcl}	% Article class of KOMA-script with 11pt font and a4 format
\usepackage[T1]{fontenc}
\usepackage{fourier}

\usepackage[english]{babel}															% English language/hyphenation
\usepackage[protrusion=true,expansion=true]{microtype}	% Better typography
\usepackage{amsmath,amsfonts,amsthm}					% Math packages
\usepackage[pdftex]{graphicx}														% Enable pdflatex
\usepackage{url}
\usepackage{caption}
\usepackage{subcaption}
\usepackage{graphicx}

% Code listing
\usepackage{listings}
\usepackage{color}

 
\definecolor{codegreen}{rgb}{0,0.6,0}
\definecolor{codegray}{rgb}{0.5,0.5,0.5}
\definecolor{codepurple}{rgb}{0.58,0,0.82}
\definecolor{backcolour}{rgb}{0.95,0.95,0.92}
 
\lstdefinestyle{mystyle}{
    backgroundcolor=\color{backcolour},   
    commentstyle=\color{codegray},
    keywordstyle=\color{magenta},
    numberstyle=\tiny\color{codegray},
    stringstyle=\color{codepurple},
    basicstyle=\small\ttfamily,
    breakatwhitespace=false,         
    breaklines=true,                 
    captionpos=b,                    
    keepspaces=true,                 
    numbers=left,                    
    numbersep=5pt,                  
    showspaces=false,                
    showstringspaces=false,
    showtabs=false,                  
    tabsize=2
}
\lstset{columns=flexible}
\lstset{keepspaces=true}



\lstset{style=mystyle}


%%% Custom sectioning (sectsty package)
\usepackage{sectsty}									% Custom sectioning (see below)
\allsectionsfont{\centering \normalfont\scshape}		% Change font of al section commands


%%% Custom headers/footers (fancyhdr package)
\usepackage{fancyhdr}
\pagestyle{fancyplain}
\fancyhead{}									% No page header
\fancyfoot[C]{}									% Empty
\fancyfoot[R]{\thepage}							% Pagenumbering
\renewcommand{\headrulewidth}{0pt}				% Remove header underlines
\renewcommand{\footrulewidth}{0pt}				% Remove footer underlines
\setlength{\headheight}{13.6pt}


%%% Equation and float numbering
\numberwithin{equation}{section}		% Equationnumbering: section.eq#
\numberwithin{figure}{section}			% Figurenumbering: section.fig#
\numberwithin{table}{section}			% Tablenumbering: section.tab#


%%% Maketitle metadata
\newcommand{\horrule}[1]{\rule{\linewidth}{#1}} 	% Horizontal rule

\title{
		%\vspace{-1in} 	
		\usefont{OT1}{bch}{b}{n}
		\normalfont \normalsize \textsc{Space Communication Systems} \\ [25pt]
		\horrule{0.5pt} \\[0.4cm]
		\huge SIMCOM: Simuation of a DVB-S communication channel \\
		\horrule{2pt} \\[0.5cm]
}
\author{
		\normalfont 								\normalsize
        Juan Pablo CUADRO, Lo\"{i}c VEILLARD\\[-3pt]	\normalsize
        \today
}
\date{}


%%% Begin document
\begin{document}
\maketitle

\section{Introduction}
In this project we created a baseband simulator in order to evaluate the performances in the DVB-S physical layer taking into account the following:

\begin{itemize}

  \item QPSK modulation with Grey coding and raised cosine pulse shaping
  \item Forward error correction (FEC)
  \item Additive white Gaussian noise (AWGN) channel \ldots

\end{itemize}

\section{DVB-S}



\section{System}

Emitter and receiver architecture are shown in Figures ~\ref{fig:TxArch} and ~\ref{fig:RxArch} respectively. Channel coding is achieved using two concatenated codes. Outer coding is implemented using a shortened Reed-Solomon code. Inner coding, on the other hand, is based on a convolutional coding scheme. The System allows for a range of punctured convolutional codes, based on a rate 1/2 convolutional code. This allows selection of the most appropriate level of error correction for a given service or data rate. supported code rates are 1/2, 2/3, 3/4, 5/6 and 7/8.

\begin{figure}[htb]
\centering
\includegraphics[scale=0.25]{EmitterL.png}
\caption{Emitter Architecture}\label{fig:TxArch}
\end{figure}

\begin{figure}[htb]
\centering
\includegraphics[scale=0.25]{ReceiverL.png}
\caption{Receiver Architecture}\label{fig:RxArch}
\end{figure}


\subsection{Outer Coding}

Outer coding is based on the shortened RS(204,188,T=8) code which in turn derives from an original RS(255,239,T=8). This shortening is achieved by adding 51 bytes, all set to zero, before the information bytes at the input of the (255,239) encoder. After the RS coding procedure these null bytes are discarded.

In order to protect transmissions from burst errors, a interleaver is added at the output of the outer encoder. Note that in this simulation a block interleaver is used while in the original DVB-S standard a convolutional interleaver. This allows the error bursts at the output of the inner decoder to be randomized in order to improve the burst error correction capabilities of the outer decoder.

\subsection{Inner Coding}

A range of punctured convolutional codes, based on a rate 1/2 convolutional code with constraint length K = 7 are possible. The selection of the most appropriate level of error correction for a given service is permitted this way. Possible codes allow with code rates of 1/2, 2/3, 3/4, 5/6 and 7/8. Generator polynomials are $g_1=171_{\text{oct}}$ and $g_2=133{\text{oct}}$ as defined in the standard. This corresponds to 64 possible trellis states. Simulations done will only include 2/3 puncturing using the puncturing matrix $P=[1 1 0 1]$. Decoding is done using a Viterbi decoder.

\subsection{Bit Mapping, Baseband Shaping and Modulation}

System uses conventional Gray-coded QPSK with absolute mapping. Prior to modulation, I and Q samples  (mathematically represented by a series of Dirac deltas, multiplied by the amplitudes I and Q, spaced by the symbol duration $T_s=1/R_s$ are square root raised cosine filtered. The roll-off factor is $\alpha=0.35$

\section{Simulation Results}

\subsection{Observed Signals}

First of all, let us take a look at the PSD of the transmitted signal (baseband). For this, Welch's method is used to estimate the power spectral density. The resulting PSD can be seen in Figure ~\ref{fig:TxPsd}.

\begin{figure}[htb]
\centering
\includegraphics[scale=0.50]{matp_TxPsd.eps}
\caption{PSD of transmitted signal. Welch blabla}\label{fig:TxPsd}
\end{figure}

Next up we can take a look at the received QPSK symbols and compare them to the transmitted ones. It is interest

\begin{figure}[htb]
\centering
\includegraphics[scale=0.50]{matp_Constellation10dB.png}
\caption{Transmitted and received QPSK symbols with $E_b/N_0 = 10\text{dB}$  }\label{fig:Constellation}
\end{figure}



\begin{figure}
\centering
\begin{subfigure}{.5\textwidth}
  \centering
  \includegraphics[width=1\linewidth]{matp_EyeClean.eps}
  \caption{No noise}
  \label{fig:eyeClean}
\end{subfigure}%
\begin{subfigure}{.5\textwidth}
  \centering
  \includegraphics[width=1\linewidth]{matp_Eye10dB.eps}
  \caption{$E_b/N_0 = 10\text{dB}$}
  \label{fig:eyeDirty}
\end{subfigure}
\caption{Eye diagrams at the output of the matched filter.}
\label{fig:test}
\end{figure}

\subsection{QPSK Performance}

In order to validate the implementation bit-error-rate (BER) performance is evaluated and compared against theoretical results. The expression for BER over an AWGN channel in QPSK is given by the following equation:

\begin{equation} \label{eq:qpskBer}
BER = Q\left(\sqrt{\frac{2E_b}{N_0}}\right)
\end{equation}

Where $E_b/N_0$ is the the energy per bit to noise power spectral density ratio. Results can be generated by bypassing all FEC blocks. A comparison between theoretical results and simulation results can be seen in Figure ~\ref{}.



%%% End document
\end{document}